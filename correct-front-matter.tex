\documentclass[]{msu-thesis}
\usepackage{lmodern}
\usepackage{amssymb,amsmath}
\usepackage{ifxetex,ifluatex}
\usepackage{fixltx2e} % provides \textsubscript
\ifnum 0\ifxetex 1\fi\ifluatex 1\fi=0 % if pdftex
  \usepackage[T1]{fontenc}
  \usepackage[utf8]{inputenc}
\else % if luatex or xelatex
  \ifxetex
    \usepackage{mathspec}
  \else
    \usepackage{fontspec}
  \fi
  \defaultfontfeatures{Ligatures=TeX,Scale=MatchLowercase}
\fi
% use upquote if available, for straight quotes in verbatim environments
\IfFileExists{upquote.sty}{\usepackage{upquote}}{}
% use microtype if available
\IfFileExists{microtype.sty}{%
\usepackage{microtype}
\UseMicrotypeSet[protrusion]{basicmath} % disable protrusion for tt fonts
}{}
\usepackage[margin=1in]{geometry}
\usepackage{hyperref}
\hypersetup{unicode=true,
            pdftitle={Spatial patterning of lake nutrients and morphometry at macroscales: importance of regional factors and aquatic-terrestrial linkages},
            pdfauthor={Joseph Jeremy Stachelek},
            pdfborder={0 0 0},
            breaklinks=true}
\urlstyle{same}  % don't use monospace font for urls
\usepackage{natbib}
\bibliographystyle{jsta}
\usepackage{longtable,booktabs}
\usepackage{graphicx,grffile}
\makeatletter
\def\maxwidth{\ifdim\Gin@nat@width>\linewidth\linewidth\else\Gin@nat@width\fi}
\def\maxheight{\ifdim\Gin@nat@height>\textheight\textheight\else\Gin@nat@height\fi}
\makeatother
% Scale images if necessary, so that they will not overflow the page
% margins by default, and it is still possible to overwrite the defaults
% using explicit options in \includegraphics[width, height, ...]{}
\setkeys{Gin}{width=\maxwidth,height=\maxheight,keepaspectratio}
\IfFileExists{parskip.sty}{%
\usepackage{parskip}
}{% else
\setlength{\parindent}{0pt}
\setlength{\parskip}{6pt plus 2pt minus 1pt}
}
\setlength{\emergencystretch}{3em}  % prevent overfull lines
\providecommand{\tightlist}{%
  \setlength{\itemsep}{0pt}\setlength{\parskip}{0pt}}
\setcounter{secnumdepth}{5}
% Redefines (sub)paragraphs to behave more like sections
\ifx\paragraph\undefined\else
\let\oldparagraph\paragraph
\renewcommand{\paragraph}[1]{\oldparagraph{#1}\mbox{}}
\fi
\ifx\subparagraph\undefined\else
\let\oldsubparagraph\subparagraph
\renewcommand{\subparagraph}[1]{\oldsubparagraph{#1}\mbox{}}
\fi

%%% Use protect on footnotes to avoid problems with footnotes in titles
\let\rmarkdownfootnote\footnote%
\def\footnote{\protect\rmarkdownfootnote}

%%% Change title format to be more compact
\usepackage{titling}

% Create subtitle command for use in maketitle
\newcommand{\subtitle}[1]{
  \posttitle{
    \begin{center}\large#1\end{center}
    }
}

%\setlength{\droptitle}{-2em}
%  \title{Understanding Work With Data in Summer STEM Programs Through An Experience Sampling Method Approach}
%  \pretitle{\vspace{\droptitle}\centering\huge}
%  \posttitle{\par}
%  \author{Joseph J. Stachelek}
%  \preauthor{\centering\large\emph}
%  \postauthor{\par}
%  \predate{\centering\large\emph}
%  \postdate{\par}
%  \date{2017-11-24}
%
\usepackage{bibentry}
\nobibliography*
\newcommand\hangbibentry[1]{%
    \smallskip\par\hangpara{1em}{1}\bibentry{#1}\smallskip\par %{indent}{afterline}
}

\usepackage{doi, hyperref}

\frontmatter
%\pagenumbering{Roman}
\newpage

\newpage

\pagebreak

\pagebreak


\usepackage{booktabs}
\usepackage{amsthm}
\makeatletter
\def\thm@space@setup{%
  \thm@preskip=8pt plus 2pt minus 4pt
  \thm@postskip=\thm@preskip
}
\makeatother
\setlength{\parindent}{4em}
\setlength{\parskip}{0em}

\title{Spatial patterning of lake nutrients and morphometry at macroscales: importance of regional factors and aquatic-terrestrial linkages}
\author{Joseph Jeremy Stachelek}
\fieldofstudy{Fisheries and Wildlife}
\dedication{This dissertation is dedicated to...}
\date{2020}

%%%%%% MSU-THESIS stuff
% \usepackage[T1]{fontenc}
% \usepackage{newtxtext,newtxmath} % If they want Times we’ll give them Times
% \usepackage{amsmath}
% \usepackage[]{natbib}
% \bibliographystyle{unified}


% If you need newlines in your title, you must use \protect\\
% \title{Understanding Work With Data in Summer STEM Programs Through An Experience Sampling Method Approach}
% \author{Joseph J. Stachelek}
% \fieldofstudy{Educational Psychology and Educational Technology}
% \dedication{This dissertation is dedicated to Katie and to Jonah, who (mostly) happily slept through most of its writing.}
% \date{2018}
\usepackage{listings}

% \lstset{language=TeX,basicstyle={\ttfamily}}
% \usepackage{lipsum}
% \usepackage{xcolor}
% \usepackage{gb4e}

%\usepackage[bookmarksopenlevel=2,bookmarks=true]{hyperref} % not needed but here for testing
%
% \counterwithin{exx}{chapter}
% \singlegloss

% Uncomment the next line for single spaced examples with gb4e
% patchcommand{\exe}{\SingleSpacing}{}

% % This code is an example of how to set up a new list of
% \newlistof{listoflistings}{lol}{List of Listings}
% \newfloat[chapter]{listing}{lol}{Listing}
% \newlistentry{listing}{lol}{0}
% \renewcommand*{\cftlistingname}{Listing\space}
% \renewcommand*{\cftlistingaftersnum}{\msucaptiondelim}

\usepackage{booktabs}
\usepackage{longtable}
\usepackage{array}
\usepackage{multirow}
\usepackage[table]{xcolor}
\usepackage{wrapfig}
\usepackage{float}
\usepackage{colortbl}
\usepackage{pdflscape}
\usepackage{tikz}
\usepackage{tabu}
\usepackage{threeparttable}

\usepackage{amsthm}
\newtheorem{theorem}{Theorem}[section]
\newtheorem{lemma}{Lemma}[section]
\theoremstyle{definition}
\newtheorem{definition}{Definition}[section]
\newtheorem{corollary}{Corollary}[section]
\newtheorem{proposition}{Proposition}[section]
\theoremstyle{definition}
\newtheorem{example}{Example}[section]
\theoremstyle{definition}
\newtheorem{exercise}{Exercise}[section]
\theoremstyle{remark}
\newtheorem*{remark}{Remark}
\newtheorem*{solution}{Solution}

\usepackage{color}
\usepackage{fancyvrb}
\newcommand{\VerbBar}{|}
\newcommand{\VERB}{\Verb[commandchars=\\\{\}]}
\DefineVerbatimEnvironment{Highlighting}{Verbatim}{commandchars=\\\{\}}
% Add ',fontsize=\small' for more characters per line
\usepackage{framed}
\definecolor{shadecolor}{RGB}{248,248,248}
\newenvironment{Shaded}{\begin{snugshade}}{\end{snugshade}}
\newcommand{\AlertTok}[1]{\textcolor[rgb]{0.94,0.16,0.16}{#1}}
\newcommand{\AnnotationTok}[1]{\textcolor[rgb]{0.56,0.35,0.01}{\textbf{\textit{#1}}}}
\newcommand{\AttributeTok}[1]{\textcolor[rgb]{0.77,0.63,0.00}{#1}}
\newcommand{\BaseNTok}[1]{\textcolor[rgb]{0.00,0.00,0.81}{#1}}
\newcommand{\BuiltInTok}[1]{#1}
\newcommand{\CharTok}[1]{\textcolor[rgb]{0.31,0.60,0.02}{#1}}
\newcommand{\CommentTok}[1]{\textcolor[rgb]{0.56,0.35,0.01}{\textit{#1}}}
\newcommand{\CommentVarTok}[1]{\textcolor[rgb]{0.56,0.35,0.01}{\textbf{\textit{#1}}}}
\newcommand{\ConstantTok}[1]{\textcolor[rgb]{0.00,0.00,0.00}{#1}}
\newcommand{\ControlFlowTok}[1]{\textcolor[rgb]{0.13,0.29,0.53}{\textbf{#1}}}
\newcommand{\DataTypeTok}[1]{\textcolor[rgb]{0.13,0.29,0.53}{#1}}
\newcommand{\DecValTok}[1]{\textcolor[rgb]{0.00,0.00,0.81}{#1}}
\newcommand{\DocumentationTok}[1]{\textcolor[rgb]{0.56,0.35,0.01}{\textbf{\textit{#1}}}}
\newcommand{\ErrorTok}[1]{\textcolor[rgb]{0.64,0.00,0.00}{\textbf{#1}}}
\newcommand{\ExtensionTok}[1]{#1}
\newcommand{\FloatTok}[1]{\textcolor[rgb]{0.00,0.00,0.81}{#1}}
\newcommand{\FunctionTok}[1]{\textcolor[rgb]{0.00,0.00,0.00}{#1}}
\newcommand{\ImportTok}[1]{#1}
\newcommand{\InformationTok}[1]{\textcolor[rgb]{0.56,0.35,0.01}{\textbf{\textit{#1}}}}
\newcommand{\KeywordTok}[1]{\textcolor[rgb]{0.13,0.29,0.53}{\textbf{#1}}}
\newcommand{\NormalTok}[1]{#1}
\newcommand{\OperatorTok}[1]{\textcolor[rgb]{0.81,0.36,0.00}{\textbf{#1}}}
\newcommand{\OtherTok}[1]{\textcolor[rgb]{0.56,0.35,0.01}{#1}}
\newcommand{\PreprocessorTok}[1]{\textcolor[rgb]{0.56,0.35,0.01}{\textit{#1}}}
\newcommand{\RegionMarkerTok}[1]{#1}
\newcommand{\SpecialCharTok}[1]{\textcolor[rgb]{0.00,0.00,0.00}{#1}}
\newcommand{\SpecialStringTok}[1]{\textcolor[rgb]{0.31,0.60,0.02}{#1}}
\newcommand{\StringTok}[1]{\textcolor[rgb]{0.31,0.60,0.02}{#1}}
\newcommand{\VariableTok}[1]{\textcolor[rgb]{0.00,0.00,0.00}{#1}}
\newcommand{\VerbatimStringTok}[1]{\textcolor[rgb]{0.31,0.60,0.02}{#1}}
\newcommand{\WarningTok}[1]{\textcolor[rgb]{0.56,0.35,0.01}{\textbf{\textit{#1}}}}

\usepackage{newfloat,tocloft}

\DeclareFloatingEnvironment[name={Figure},fileext=lsf,listname={List of Supplementary Figures}]{suppfigure}
\DeclareCaptionLabelFormat{suppfigure}{#1.#2}
\captionsetup[suppfigure]{labelformat=suppfigure}
\renewcommand{\thesuppfigure}{\arabic{suppfigure}}

\DeclareFloatingEnvironment[name={Table},fileext=lst,listname={List of Supplementary Tables}]{supptable}
\DeclareCaptionLabelFormat{supptable}{#1.#2}
\captionsetup[supptable]{labelformat=supptable}
\renewcommand{\thesupptable}{\arabic{supptable}}

\begin{document}
%\maketitle

\maketitlepage
% Next make the abstract
\begin{abstract}
% Your abstract goes here.  Master's 1 page max. PhD 2 page max.
Lakes are classically viewed as discrete ecosystems bounded on all sides by land. However, a narrow focus on lakes as discrete units is incompatible with the scale of many management programs and ignores the placement of lakes relative to their larger ecological context. While it is clear that lakes are not isolated units but are instead embedded components of lake-river networks and have a broader landscape (i.e. regional) context, it is not always clear how this embedding is borne out quantitatively. For example, is the position of a lake in a multi-lake network a dominant predictor of nutrient retention? Or, how strongly does the arrangement of streams and near-stream land-use (i.e. aquatic-terrestrial linkages) affect lake nutrient concentrations? To date, we have been unable to quantitatively address these questions in a data-driven manner because the necessary data has not previously been available for many lakes over large geographic extents (i.e. the macroscale). As a result, prior research has mostly been conducted on single lakes or in some cases groups of nearby lakes in a single region. In the following chapters, I develop several macroscale lake databases that include hundreds to thousands of lakes across diverse local and regional settings. I use these databases to investigate the roles of both local and regional processes and aquatic-terrestrial linkages in determining lake nutrient retention, nutrient concentrations, and basic lake morphometry.

In my first chapter, I show that throughout Northeastern and Midwest US, lakes with higher connectivity have lower nutrient retention but this “connectivity effect” is apparent at the scale of entire lake networks rather than more localized lake subwatersheds. My findings suggest that a broader whole-network perspective is likely to be more effective than a narrow lake-specific perspective in regulatory frameworks focused on eutrophication. In my second chapter, I show that lake nutrient concentrations are related to a variety of agricultural activity measures beyond the percent of the watershed in agricultural land use. I show that one measure in particular, the percentage of agricultural land use in near-stream areas, when elevated signals a high likelihood of elevated lake nutrient concentrations. I further show that lake total phosphorus concentrations have different relationships with measures of agricultural activity relative to lake total nitrogen concentrations. My findings suggest that dual nutrient control strategies in agricultural landscapes are likely to be more effective at controlling lake eutrophication relative to single nutrient control strategies. In my third and final chapter, I test a conceptual model substituting rarely available bathymetry-derived measures of lake geometry with widely available GIS-derived measures of lake geometry that are available for all lakes. I show that lake depth prediction is more sensitive to imperfect measures of the vertical (i.e. slope) dimension of lakes rather than imperfect horizontal measures. My findings suggest that prior estimates of lake depth likely overestimate the depth of specific types of lakes such as those with concave cross-section shapes and those classified as reservoirs.
\end{abstract}

% Force a newpage
\clearpage
% Make the copyright page. The Graduate School ridiculously prohibits you
% from having a copyright page unless you pay ProQuest to register the copyright.
% This should be illegal, but I didn't make up the rule.

\makecopyrightpage

% If you have a dedication page, uncomment the next command to print the dedication page
%
% \makededicationpage
%
% \clearpage

% Your Acknowledgments are formatted like a chapter, but with no number
\chapter*{Acknowledgments}
\DoubleSpacing % Acknowledgements should be double spaced
It’s cliche to say you’ve stood on the shoulders of giants but maybe it’s just this type of elevated perspective that’s needed to see how the pieces fit together - to see how lakes are related to each other - and to see how lakes are related to their ecological context. The shoulders I’ve stood on include of course my advisor, Pat Soranno, as well as the other members of our lab Kendra Cheruvelil, Katelyn King, Ian McCullough, Autumn Poisson, Nick Skaff, and Nicole Smith whose support has been much appreciated. I would also like to thank my remaining committee members Drs Elise Zipkin and Dana Infante who provided excellent advice and direction on this Dissertation journey. As a group, I would like to thank the members of the Continental Limnology and CNH Lakes Projects whose diversity of perspectives provided much needed context for my research. Also, thank you to everyone from the numerous  federal, state, tribal, and non-profit agencies, as well as university researchers and citizen scientist data collectors and curators whose data unpin my research. Finally, I would like to thank Sheena Stachelek and the rest of my family for their love and encouragement.

\vspace{2em}
\noindent
“You can’t stop water. Water will find a way.” \\
- Colin Saunders

\clearpage

\chapter*{Preface}
\DoubleSpacing
The chapters in this dissertation have been written as standalone papers with the following citations:
\vspace{0.3em}

\SingleSpacing
\begin{itemize}
\item \hangbibentry{stachelek_2019}

\item \hangbibentry{stachelekGranularMeasuresAgricultural2020}

\item \hangbibentry{stachelekBathymetryDataThousandsInPrep}
\end{itemize}
\DoubleSpacing

\vspace{0.1em}
\noindent
In addition to the papers that make up my Dissertation, I was an author on papers that are not explicitly listed above but were instead precursors to dissertation chapters (Stachelek and Sorrano (2019) and Stachelek et al. (In prep) respectively):
\vspace{0.3em}

\SingleSpacing
\begin{itemize}
\item \hangbibentry{stachelekNationalEutrophicationSurvey2017}

\item \hangbibentry{hollisterLakemorphoCalculatingLake2017}
\end{itemize}
\DoubleSpacing

\vspace{0.1em}
\noindent
I authored other papers that were not directly related to my dissertation but were instead a result of my participation with two collaborative research groups, the Continental Limnology project (\href{https://lagoslakes.org/projects/continental-limnology}{https://lagoslakes.org/projects/continental-limnology}) and the CNH Lakes project (\href{https://www.cnhlakes.frec.vt.edu/}{https://www.cnhlakes.frec.vt.edu/}). The citation for each of these papers and my contribution to each is listed below. For the Continental Limnology project, these are:

\vspace{0.3em}

\begin{itemize}
\SingleSpacing

\item \hangbibentry{wagnerIncreasingAccuracyLake2019}

\item \hangbibentry{collinsWinterPrecipitationSummer2019}

\item \hangbibentry{mcculloughLakesFeelBurn2019a}

\item \hangbibentry{mcculloughApplyingPatchmatrixModel2019}

\item \hangbibentry{qianImplicationsSimpsonParadox2019}

\item \hangbibentry{sorannoLAGOSNEMultiscaledGeospatial2017}

\end{itemize}
\DoubleSpacing

\vspace{0.1em}
\noindent
For the CNH Lakes project, these are:
\vspace{0.3em}

\SingleSpacing
\begin{itemize}
\item \hangbibentry{wardIntegratingFastSlow2018}

\item \hangbibentry{cobournConceptPracticePolicy2018}
\end{itemize}
\DoubleSpacing

\vspace{0.1em}
\noindent
In addition to papers, I authored the following software packages:
\vspace{0.3em}

\SingleSpacing
\begin{itemize}

\item \hangbibentry{R-nhdR}

\item \hangbibentry{stachelekLAGOSNEInterfaceLake2019}

\item \hangbibentry{stachelekGssurgoPythonToolbox2019}

\end{itemize}
\DoubleSpacing

\clearpage

% We need to turn single spacing back on for the contents/figures/tables lists
\SingleSpacing
\tableofcontents* % table of contents will not be listed in the TOC
\clearpage
\listoftables % comment this out if you have no tables
% \listofsupptables
\clearpage
\listoffigures % comment this out if you have no figures
% \listofsuppfigures
\mainmatter
% If you have a list of abbreviations/symbols it would go here preceded by a \clearpage
% See the class documentation and the Memoir manual for how to create other lists
%

\chapter*{Introduction}\stepcounter{chapter}\addcontentsline{toc}{chapter}{Introduction}
\setcounter{chapter}{0}

Lakes are classically viewed as discrete ecosystems bounded on all sides by land. This framing has served as a useful organizing principle particularly with regard to the study of specific waterbodies. However, a narrow focus on lakes as discrete units is incompatible with the scale of many management programs and ignores the links between lakes and their larger ecological context \citep{likensLinkagesTerrestrialAquatic1974}. In recognizing these limitations, a specific focus of modern limnology has been on further integrating ecological context in the study of lakes. Notable concepts and frameworks developed as part of these efforts include aquatic-terrestrial linkages \citep{likensLinkagesTerrestrialAquatic1974}, the freshwater continuum \citep{vannoteRiverContinuumConcept1980}, landscape limnology \citep{sorannoUsingLandscapeLimnology2010}, and macrosystems ecology \citep{heffernanMacrosystemsEcologyUnderstanding2014}. \par

Despite much progress, several limiting factors have prevented a more full consideration of lakes relative to their ecological context. Foremost of these has been limited data availability.  It is common for studies to consider the ecological context of lakes relative to their watersheds for a single lake or several lakes within a single geographic region. However, it is far less common for studies to consider many lakes across multiple regions (i.e. at macroscales) where distances span hundreds to thousands of kilometers (but see \citet{usepa1975}, \citet{landers1988eastern}, and \citet{usepa_nla2012}). A consequence of the mostly narrow focus on relatively few lakes is that we lack the perspective necessary to formulate general relationships between lakes and their ecological context that hold beyond a single lake or region. As macroscale lake datasets become available, such as \citep{sorannoLAGOSNEMultiscaledGeospatial2017, williamsDatabaseGeoreferencedNutrient2017, readWaterQualityData2017}, we now have the ability to test the generality of fine-scale relationships identified in prior studies at the macroscale.

A further limiting factor preventing a fuller consideration of lake ecological context has been lack of data analysis techniques that both account for and leverage spatial variability at broad scales. Such techniques are necessary to account for multiple interacting drivers of the lake ecological status at such scales \citep{allanLandscapesRiverscapesInfluence2004}. A key technique, which I use throughout the following chapters, is to build models featuring regionally varying coefficients whereby different relationships between lake quantities and landscape predictors are estimated for different geographic regions. This avoids some of the risk of drawing imprecise or incorrect conclusions due to lumping together lakes with fundamentally different responses to a given predictor variable (Qian et al. 2019). Furthermore, this technique allowed for a richer post-hoc examination of model results in light of unobserved or unaccounted for regional variation in underlying lake drivers. \par

My aim was to more rigorously test conventional understanding of lakes gained from studies at limited spatial extents to a more diverse set of lakes representative of continental-scale gradients through scaling and extrapolation. Therefore, I asked two overarching questions: 1) What is the role of local and regional processes and aquatic-terrestrial linkages in determining lake nutrient retention, nutrient concentrations, and basic lake morphometry? and 2) How do relationships between lake nutrient retention, nutrient concentrations and morphology differ when computed over different extents and with different levels of detail?

\chapter{Does freshwater connectivity influence phosphorus retention in lakes?}\label{intro}

\DoubleSpacing
