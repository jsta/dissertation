\documentclass[]{msu-thesis}
\usepackage{lmodern}
\usepackage{amssymb,amsmath}
\usepackage{ifxetex,ifluatex}
\usepackage{fixltx2e} % provides \textsubscript
\ifnum 0\ifxetex 1\fi\ifluatex 1\fi=0 % if pdftex
  \usepackage[T1]{fontenc}
  \usepackage[utf8]{inputenc}
\else % if luatex or xelatex
  \ifxetex
    \usepackage{mathspec}
  \else
    \usepackage{fontspec}
  \fi
  \defaultfontfeatures{Ligatures=TeX,Scale=MatchLowercase}
\fi
% use upquote if available, for straight quotes in verbatim environments
\IfFileExists{upquote.sty}{\usepackage{upquote}}{}
% use microtype if available
\IfFileExists{microtype.sty}{%
\usepackage{microtype}
\UseMicrotypeSet[protrusion]{basicmath} % disable protrusion for tt fonts
}{}
\usepackage[margin=1in]{geometry}
\usepackage{hyperref}
\hypersetup{unicode=true,
            pdftitle={Spatial patterning of lake nutrients and morphometry at macroscales: importance of regional factors and aquatic-terrestrial linkages},
            pdfauthor={Joseph J. Stachelek},
            pdfborder={0 0 0},
            breaklinks=true}
\urlstyle{same}  % don't use monospace font for urls
\usepackage{natbib}
\bibliographystyle{apalike}
\usepackage{longtable,booktabs}
\usepackage{graphicx,grffile}
\makeatletter
\def\maxwidth{\ifdim\Gin@nat@width>\linewidth\linewidth\else\Gin@nat@width\fi}
\def\maxheight{\ifdim\Gin@nat@height>\textheight\textheight\else\Gin@nat@height\fi}
\makeatother
% Scale images if necessary, so that they will not overflow the page
% margins by default, and it is still possible to overwrite the defaults
% using explicit options in \includegraphics[width, height, ...]{}
\setkeys{Gin}{width=\maxwidth,height=\maxheight,keepaspectratio}
\IfFileExists{parskip.sty}{%
\usepackage{parskip}
}{% else
\setlength{\parindent}{0pt}
\setlength{\parskip}{6pt plus 2pt minus 1pt}
}
\setlength{\emergencystretch}{3em}  % prevent overfull lines
\providecommand{\tightlist}{%
  \setlength{\itemsep}{0pt}\setlength{\parskip}{0pt}}
\setcounter{secnumdepth}{5}
% Redefines (sub)paragraphs to behave more like sections
\ifx\paragraph\undefined\else
\let\oldparagraph\paragraph
\renewcommand{\paragraph}[1]{\oldparagraph{#1}\mbox{}}
\fi
\ifx\subparagraph\undefined\else
\let\oldsubparagraph\subparagraph
\renewcommand{\subparagraph}[1]{\oldsubparagraph{#1}\mbox{}}
\fi

%%% Use protect on footnotes to avoid problems with footnotes in titles
\let\rmarkdownfootnote\footnote%
\def\footnote{\protect\rmarkdownfootnote}

%%% Change title format to be more compact
\usepackage{titling}

% Create subtitle command for use in maketitle
\newcommand{\subtitle}[1]{
  \posttitle{
    \begin{center}\large#1\end{center}
    }
}

%\setlength{\droptitle}{-2em}
%  \title{Understanding Work With Data in Summer STEM Programs Through An Experience Sampling Method Approach}
%  \pretitle{\vspace{\droptitle}\centering\huge}
%  \posttitle{\par}
%  \author{Joseph J. Stachelek}
%  \preauthor{\centering\large\emph}
%  \postauthor{\par}
%  \predate{\centering\large\emph}
%  \postdate{\par}
%  \date{2017-11-24}
%
\usepackage{bibentry}
\nobibliography*
\newcommand\hangbibentry[1]{%
    \smallskip\par\hangpara{1em}{1}\bibentry{#1}\smallskip\par %{indent}{afterline}
}



\frontmatter
%\pagenumbering{Roman}
\newpage

\newpage

\pagebreak

\pagebreak


\usepackage{booktabs}
\usepackage{amsthm}
\makeatletter
\def\thm@space@setup{%
  \thm@preskip=8pt plus 2pt minus 4pt
  \thm@postskip=\thm@preskip
}
\makeatother
\setlength{\parindent}{4em}
\setlength{\parskip}{0em}

\title{Spatial patterning of lake nutrients and morphometry at macroscales: importance of regional factors and aquatic-terrestrial linkages}
\author{Joseph J. Stachelek}
\fieldofstudy{Fisheries and Wildlife}
\dedication{This dissertation is dedicated to...}
\date{2020}

%%%%%% MSU-THESIS stuff
% \usepackage[T1]{fontenc}
% \usepackage{newtxtext,newtxmath} % If they want Times we’ll give them Times
% \usepackage{amsmath}
% \usepackage[]{natbib}
% \bibliographystyle{unified}


% If you need newlines in your title, you must use \protect\\
% \title{Understanding Work With Data in Summer STEM Programs Through An Experience Sampling Method Approach}
% \author{Joseph J. Stachelek}
% \fieldofstudy{Educational Psychology and Educational Technology}
% \dedication{This dissertation is dedicated to Katie and to Jonah, who (mostly) happily slept through most of its writing.}
% \date{2018}
\usepackage{listings}

% \lstset{language=TeX,basicstyle={\ttfamily}}
% \usepackage{lipsum}
% \usepackage{xcolor}
% \usepackage{gb4e}

%\usepackage[bookmarksopenlevel=2,bookmarks=true]{hyperref} % not needed but here for testing
%
% \counterwithin{exx}{chapter}
% \singlegloss

% Uncomment the next line for single spaced examples with gb4e
% patchcommand{\exe}{\SingleSpacing}{}

% % This code is an example of how to set up a new list of
% \newlistof{listoflistings}{lol}{List of Listings}
% \newfloat[chapter]{listing}{lol}{Listing}
% \newlistentry{listing}{lol}{0}
% \renewcommand*{\cftlistingname}{Listing\space}
% \renewcommand*{\cftlistingaftersnum}{\msucaptiondelim}

\usepackage{booktabs}
\usepackage{longtable}
\usepackage{array}
\usepackage{multirow}
\usepackage[table]{xcolor}
\usepackage{wrapfig}
\usepackage{float}
\usepackage{colortbl}
\usepackage{pdflscape}
\usepackage{tikz}
\usepackage{tabu}
\usepackage{threeparttable}

\usepackage{amsthm}
\newtheorem{theorem}{Theorem}[section]
\newtheorem{lemma}{Lemma}[section]
\theoremstyle{definition}
\newtheorem{definition}{Definition}[section]
\newtheorem{corollary}{Corollary}[section]
\newtheorem{proposition}{Proposition}[section]
\theoremstyle{definition}
\newtheorem{example}{Example}[section]
\theoremstyle{definition}
\newtheorem{exercise}{Exercise}[section]
\theoremstyle{remark}
\newtheorem*{remark}{Remark}
\newtheorem*{solution}{Solution}

\usepackage{color}
\usepackage{fancyvrb}
\newcommand{\VerbBar}{|}
\newcommand{\VERB}{\Verb[commandchars=\\\{\}]}
\DefineVerbatimEnvironment{Highlighting}{Verbatim}{commandchars=\\\{\}}
% Add ',fontsize=\small' for more characters per line
\usepackage{framed}
\definecolor{shadecolor}{RGB}{248,248,248}
\newenvironment{Shaded}{\begin{snugshade}}{\end{snugshade}}
\newcommand{\AlertTok}[1]{\textcolor[rgb]{0.94,0.16,0.16}{#1}}
\newcommand{\AnnotationTok}[1]{\textcolor[rgb]{0.56,0.35,0.01}{\textbf{\textit{#1}}}}
\newcommand{\AttributeTok}[1]{\textcolor[rgb]{0.77,0.63,0.00}{#1}}
\newcommand{\BaseNTok}[1]{\textcolor[rgb]{0.00,0.00,0.81}{#1}}
\newcommand{\BuiltInTok}[1]{#1}
\newcommand{\CharTok}[1]{\textcolor[rgb]{0.31,0.60,0.02}{#1}}
\newcommand{\CommentTok}[1]{\textcolor[rgb]{0.56,0.35,0.01}{\textit{#1}}}
\newcommand{\CommentVarTok}[1]{\textcolor[rgb]{0.56,0.35,0.01}{\textbf{\textit{#1}}}}
\newcommand{\ConstantTok}[1]{\textcolor[rgb]{0.00,0.00,0.00}{#1}}
\newcommand{\ControlFlowTok}[1]{\textcolor[rgb]{0.13,0.29,0.53}{\textbf{#1}}}
\newcommand{\DataTypeTok}[1]{\textcolor[rgb]{0.13,0.29,0.53}{#1}}
\newcommand{\DecValTok}[1]{\textcolor[rgb]{0.00,0.00,0.81}{#1}}
\newcommand{\DocumentationTok}[1]{\textcolor[rgb]{0.56,0.35,0.01}{\textbf{\textit{#1}}}}
\newcommand{\ErrorTok}[1]{\textcolor[rgb]{0.64,0.00,0.00}{\textbf{#1}}}
\newcommand{\ExtensionTok}[1]{#1}
\newcommand{\FloatTok}[1]{\textcolor[rgb]{0.00,0.00,0.81}{#1}}
\newcommand{\FunctionTok}[1]{\textcolor[rgb]{0.00,0.00,0.00}{#1}}
\newcommand{\ImportTok}[1]{#1}
\newcommand{\InformationTok}[1]{\textcolor[rgb]{0.56,0.35,0.01}{\textbf{\textit{#1}}}}
\newcommand{\KeywordTok}[1]{\textcolor[rgb]{0.13,0.29,0.53}{\textbf{#1}}}
\newcommand{\NormalTok}[1]{#1}
\newcommand{\OperatorTok}[1]{\textcolor[rgb]{0.81,0.36,0.00}{\textbf{#1}}}
\newcommand{\OtherTok}[1]{\textcolor[rgb]{0.56,0.35,0.01}{#1}}
\newcommand{\PreprocessorTok}[1]{\textcolor[rgb]{0.56,0.35,0.01}{\textit{#1}}}
\newcommand{\RegionMarkerTok}[1]{#1}
\newcommand{\SpecialCharTok}[1]{\textcolor[rgb]{0.00,0.00,0.00}{#1}}
\newcommand{\SpecialStringTok}[1]{\textcolor[rgb]{0.31,0.60,0.02}{#1}}
\newcommand{\StringTok}[1]{\textcolor[rgb]{0.31,0.60,0.02}{#1}}
\newcommand{\VariableTok}[1]{\textcolor[rgb]{0.00,0.00,0.00}{#1}}
\newcommand{\VerbatimStringTok}[1]{\textcolor[rgb]{0.31,0.60,0.02}{#1}}
\newcommand{\WarningTok}[1]{\textcolor[rgb]{0.56,0.35,0.01}{\textbf{\textit{#1}}}}

\begin{document}
%\maketitle

\maketitlepage
% Next make the abstract
\begin{abstract}
% Your abstract goes here.  Master's 1 page max. PhD 2 page max.
Lakes are classically viewed as discrete ecosystems bounded on all sides by land. This view of lakes has persisted despite much research into the role of lakes in the freshwater continuum and the recognition that lake nutrient budgets and many other factors are affected by both local and regional processes. While it is clear that lakes are not isolated units but are instead embedded components of lake-river networks and have a broader landscape context, it is not always clear how this embedding is borne out quantitatively. For example, is the position of a lake in a multi-lake network a dominant predictor of nutrient retention? Or, how strongly does the arrangement of streams relative to watershed components affect lake nutrient concentrations? One of the reasons why we lack quantitative evidence addressing these types of questions is that previous research has mostly been conducted on single lakes or in some cases groups of nearby lakes in a single region. Rather, what is needed to address these knowledge gaps is comprehensive and broad-scale analyses of hundreds to thousands of lakes across diverse local and regional settings. To address these knowledge gaps, I investigated the roles of both local and regional processes and aquatic-terrestrial linkages in determining lake nutrient retention, nutrient concentrations, and basic lake morphometry using broadscale datasets. I have three components to my dissertation research.
First, I examined how the number and types of physical connections influence lake nutrient retention. I show that throughout Northeastern and Midwest US, lakes with higher connectivity have lower nutrient retention but this “connectivity effect” is apparent at the scale of entire lake networks rather than more localized lake subwatersheds. My findings suggest that a broader whole-network perspective is likely to be more effective than a narrow lake-specific perspective in regulatory frameworks focused on eutrophication. Second, I examined how lake nutrient concentrations are related to a variety of agricultural activity measures beyond the percent of the watershed in agricultural land use. I show that one measure in particular, the lack of riparian buffer signals a high likelihood of elevated lake nutrient concentrations. I further show that lake total phosphorus concentrations have different relationships with measures of agricultural activity relative to lake total nitrogen concentrations. My findings suggest that dual nutrient control strategies in agricultural landscapes should consider such relationships to control eutrophication. Third, I examined how lake depth is related to lake and watershed characteristics particularly those that can only be calculated from bathymetry data. I use high resolution bathymetry data to show that lake maximum depth is more reliable than mean depth, depth prediction error is sensitive to imprecise measurements of slope, and the depth of U-shaped lakes is overestimated not because they have steeper slopes but because they are shallower. Furthermore I show that lake cross section shape is more strongly related to proxies for lake hydrology rather than the slope of the land surrounding a lake. These findings underscore the fact that lake depth as it is used in predictive models of lake nutrients serve as a stand-in for otherwise missing hydrology information.
Taken together, my research points to the importance of hydrology in determining lake nutrient retention and concentrations. As a result, I see the next frontier in lake nutrient modelling as the challenge of incorporating more detailed, accurate, and fine scale measures of hydrology. This is something that is somewhat common in studies at limited spatial extents but can become intractable at broad spatial extents particularly if the output of hydrology models is excluded or unavailable.
\end{abstract}

% Force a newpage
\clearpage
% Make the copyright page. The Graduate School ridiculously prohibits you
% from having a copyright page unless you pay ProQuest to register the copyright.
% This should be illegal, but I didn't make up the rule.

\makecopyrightpage

% If you have a dedication page, uncomment the next command to print the dedication page
%
% \makededicationpage
%
% \clearpage

% Your Acknowledgments are formatted like a chapter, but with no number
\chapter*{Acknowledgments}
\DoubleSpacing % Acknowledgements should be double spaced
I'd like to thank these people who made this Dissertation possible:

Pat
DLLimno

Contlimno
CNH

Sheena

\clearpage

\chapter*{Preface}
\DoubleSpacing
The chapters in this dissertation have been written as standalone papers with the following citations:
\vspace{0.3em}

\SingleSpacing
\begin{itemize}
\item \hangbibentry{stachelek_2019}

\item \hangbibentry{stachelekjosephAgriculturalLanduseLake2020}

\item \hangbibentry{stachelekjosephBathymetryDataThousands2020}
\end{itemize}
\DoubleSpacing

\vspace{0.3em}
\noindent
In addition to the papers that make up my Dissertation, I was an author on numerous  papers that are not explicit Dissertation chapters listed above. Two of these papers were precursors to dissertation chapters (Stachelek and Sorrano (2019) and Stachelek et al. (In prep) respectively).
\vspace{0.3em}

\SingleSpacing
\begin{itemize}
\item \hangbibentry{stachelek_national_2018}

\item \hangbibentry{hollisterLakemorphoCalculatingLake2017}
\end{itemize}
\DoubleSpacing

\vspace{0.3em}
\noindent
Other papers were not directly related to my dissertation but were instead a result of my participation with two collaborative research groups, the Continental Limnology project (https://lagoslakes.org/projects/continental-limnology/) and the CNH Lakes project (https://www.cnhlakes.frec.vt.edu/). The citation for each of these papers and my contribution to each is listed below. My co-authored papers with the Continental Limnology project include:
\vspace{0.3em}

\begin{itemize}
\SingleSpacing
\item \hangbibentry{collinsWinterPrecipitationSummer2019}

\item \hangbibentry{mcculloughApplyingPatchmatrixModel2019}

\item \hangbibentry{mcculloughLakesFeelBurn2019a}

\item \hangbibentry{qianImplicationsSimpsonParadox2019}
\end{itemize}
\DoubleSpacing

\vspace{0.3em}
\noindent
Whereas my co-authored papers with the CNH Lakes project include:
\vspace{0.3em}

\SingleSpacing
\begin{itemize}
\item \hangbibentry{wardIntegratingFastSlow2018}

\item \hangbibentry{cobournConceptPracticePolicy2018}
\end{itemize}
\DoubleSpacing

\vspace{0.3em}
\noindent
In addition to papers, I authored the following software packages:
\vspace{0.3em}

\SingleSpacing
\begin{itemize}

\item \hangbibentry{R-nhdR}

\end{itemize}
\DoubleSpacing

\clearpage

% We need to turn single spacing back on for the contents/figures/tables lists
\SingleSpacing
\tableofcontents* % table of contents will not be listed in the TOC
\clearpage
\listoftables % comment this out if you have no tables
\clearpage
\listoffigures % comment this out if you have no figures
\mainmatter
% If you have a list of abbreviations/symbols it would go here preceded by a \clearpage
% See the class documentation and the Memoir manual for how to create other lists
%

\chapter*{Introduction}\stepcounter{chapter}\addcontentsline{toc}{chapter}{Introduction}
\setcounter{chapter}{0}

\chapter{Does freshwater connectivity influence phosphorus retention in lakes?}\label{intro}

\DoubleSpacing
