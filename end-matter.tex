\chapter*{Research Frontiers}
\addcontentsline{toc}{chapter}{Research Frontiers}

Every study leaves some questions unanswered or generates new lines of future inquiry. The preceding chapters generated a number of questions that represent emerging research frontiers in macroscale lake research.

For example, in Chapter 1, I used lake connectivity metrics to quantify the effect of connectivity among lakes and streams as well as connectivity of lakes and their terrestrial watersheds on lake P retention. Unfortunately, “connectivity” is a somewhat nebulous idea that is difficult to quantify with specific metrics. For instance, I found that not all metrics can be mapped onto a “high connectivity” versus “low connectivity” gradient. A potential alternative to this messy mapping of connectivity onto discrete metrics is to compare the drainage pattern of a given stream network against an optimal drainage pattern which has the highest possible connectivity for a given watershed (citation). By reporting differences from optimal (rather than raw connectivity metric values), this would allow for a more direct comparison among watersheds and between different connectivity metrics. A recent tool, the OCnet R package, may facilitate exactly such calculations as it can calculate the optimal distributions of upstream and downstream lengths, contributing area, and the space-filling attributes of specific watersheds (OCnet citation).

In Chapter 2, I examined relationships between lake nutrient concentrations and measures of agricultural activity quantified at varying levels of spatial and process-level detail. This effort was limited in the types of agricultural measures that were considered owing to the spatial extent of the study, challenges in data integration of diverse data types, and complicated model interpretation and building. For instance, it would have been great to have included other aspects of agriculture like animal feeding operations. Unfortunately, the limited spatial resolution of this data (only to county level) made for difficult integration with crop and nutrient input data. In addition, it would have been great to have modelled temporally varying lake nutrient-agriculture relationships. This is a topic that I explore In the Discussion section of Chapter 2 where I lay out some ways in which uncaptured temporal variation may have contributed to overall model uncertainty. Unfortunately, few lakes have adequate nutrient concentration time series data to drive such a model and assembly of agricultural covariate time series would have been a significant challenge.
